\documentclass{article}

% Language setting
% Replace `english' with e.g. `spanish' to change the document language
\usepackage[english]{babel}

% Set page size and margins
% Replace `letterpaper' with `a4paper' for UK/EU standard size
\usepackage[letterpaper,top=2cm,bottom=2cm,left=3cm,right=3cm,marginparwidth=1.75cm]{geometry}

% Useful packages
\usepackage{amsmath}
\usepackage{graphicx}
\usepackage[colorlinks=true, allcolors=blue]{hyperref}

\title{Documentation Projet BD}
\author{}

\begin{document}
\maketitle


\section{Analyse du problème}

\subsection*{Dépendances fonctionnelles}
On fait le choix d'inclure le numéro de téléphone et l'adresse email dans les coordonnées de contact du producteur.\\

idProducteur $\xrightarrow{}$ nom, adresse, geolocalisation, email, tel\\

On ajoute le mode de disponibilité (en stock ou sur commande)\\
idProduit $\xrightarrow{}$ nom, categorie, description, idProducteur, caracteristiques, modeDisponibilite\\

On fait le choix d'identifier la réception d'un stock par quatre valeurs : idProduit,  modeConditionnement, Poids et dateReception. En effet, on peut recevoir plusieurs types de conditionnement, et plusieurs sachets de différents poids un même jour.\\
idProduit,  modeConditionnement, Poids, dateReception $\xrightarrow{}$ Prix, Stock, datePeremption, typeDatePeremption\\

idContenant $\xrightarrow{}$ type, capacite, stock, prix, reutilisable\\

On identifie les pertes par sa date, et l'arrivage concerné, en prenant les mêmes identifiants de l'arrivage\\
datePerte, idProduit, modeConditionnement, Poids, dateReception $\xrightarrow{}$ naturePerte, Quantite\\

idClient $\xrightarrow{}$ nom, prenom, email, tel\\
idCommande $\xrightarrow{}$ idClient, dateCom, heure, statut, modePaiement, modeRecuperation\\
Dans la commande d'une livraison, on rajoute l'adresse de livraison car un client peut avoir plusieurs adresses\\
idCommandeLivraison $\xrightarrow{}$ paysLivraison, poids, distance, adresseLivraison, Prix\\
idCommandeBoutique $\xrightarrow{}$ Prix\\
\subsection*{Contraintes de multiplicité}

idProducteur $->>$ typeActivite\\
idProduit $->>$ dateDebut, dateFin\\
idClient $->>$ adressePostale\\
idCommande $->>$ idProduit, modeConditionnement, quantite, prixUnitaire, sousTotal

\subsection*{Contraintes de valeur}
Prix $>$ 0\\
Stock $>$ 0\\
Quantite $>$ 0\\
prixUnitaire $>$ 0\\
sousTotal $>$ 0\\  
modeConditionnement $\in \{"Vrac", "Pre-conditionne"$\}\\
typeDatePeremption $\in \{"DLC", "DLUO"\}$\\
statut $\in \{"En~preparation", "Prete", "En~livraison", "Livree", "Annulee"\}$\\


\end{document}
